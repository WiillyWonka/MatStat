%% -*- coding: utf-8 -*-
\documentclass[12pt,a4paper]{scrartcl} 
\usepackage[utf8]{inputenc}
\usepackage[english,russian]{babel}
\usepackage{indentfirst}
\usepackage{misccorr}
\usepackage{graphicx}
\usepackage{amsmath}
\usepackage{float}

\usepackage{xcolor}
\usepackage{hyperref}
\hypersetup{colorlinks,
  pdftitle={The title of your document},
  pdfauthor={Your name},
  allcolors=[RGB]{000 000 000}}

\begin{document}
\begin{titlepage}
  \begin{center}

    Санкт-Петербургский политехнический университет Петра Великого

    \vspace{0.25cm}
    
    Институт прикладной математики и механики
    
    Кафедра «Прикладная математика»
    \vfill

	\vspace{0.25cm}
	    Отчёт\\
	по лабораторной работе №1\\
	по дисциплине\\
	«Математическая статистика»

  \bigskip

\end{center}
\vfill

\newlength{\ML}
\settowidth{\ML}{«\underline{\hspace{0.7cm}}» \underline{\hspace{2cm}}}
\hfill\begin{minipage}{0.4\textwidth}
  Выполнил студент\\ В.\,А.~Рыженко\\
\end{minipage}%
\bigskip

\hfill\begin{minipage}{0.4\textwidth}
  Проверил:\\
к.ф.-м.н., доцент\\
Баженов Александр Николаевич\\
\end{minipage}%
\vfill

\begin{center}
  Санкт-Петербург, 2020 г.
\end{center}
\end{titlepage}

\tableofcontents
\listoffigures
\newpage

\section{Постановка задачи}
 
Для 5 распределений:
\begin{itemize}
 \item Нормальное распределение N(x, 0, 1)
 \item Распределение Коши C(x, 0, 1)
 \item Распределение Лапласа L(x, 0, $\frac{1}{\sqrt2}$)
 \item Постановка задач исследованияРаспределение Пуассона P(k, 10)
 \item Равномерное распределение U(x, $-\sqrt{3}, \sqrt{3}$) 
\end{itemize}
 
Сгенерировать выборки размером 10, 100 и 1000 элементов.
Для каждой выборки вычислить следующие статистические характеристики положения данных:
$\overline x ~\eqref{eq:mean}, med\, x~\eqref{eq:med}, z_R ~\eqref{eq:semisum}, z_Q ~\eqref{eq:semiquartile}, z_tr~\eqref{eq:trim}$. Повторить такие
вычисления 1000 раз для каждой выборки и найти среднее характеристик положения и их квадратов:

\begin{equation}\label{eq:expectation}
\centering
 E(z) = \overline z
\end{equation}

Вычислить оценку дисперсии по формуле:

\begin{equation}\label{eq:dispertion}
\centering
 D(z) = \overline {z^2} - \overline z^2
\end{equation}

Представить полученные данные в виде таблиц.

\section{Теория}
\subsection{Распределения}

\begin{itemize}
\begin{item}
Нормальное распределение
\begin{equation}\label{eq:Normal}
\centering
 N(x, 0, 1) = \frac{1}{\sqrt{2\pi} } e^{-\frac{x^2}2}
\end{equation}
\end{item}

\begin{item}
Распределение Коши
\begin{equation}\label{eq:Cauchy}
\centering
 C(x, 0, 1) = \frac{1}{\pi} \frac{1}{x^2 + 1}
\end{equation}
\end{item}

\begin{item}
Распределение Лапласа
\begin{equation}\label{eq:Laplace}
\centering
L(x, 0, \frac{1}{\sqrt2}) = \frac{1}{\sqrt{2} } e^{\sqrt2|x|}
\end{equation}
\end{item}

\begin{item}
Распределение Пуассона
\begin{equation}\label{eq:Poisson}
\centering
P(k, 10) = \frac{10^k}{k! } e^{-10}
\end{equation}
\end{item}

\begin{item}
Равномерное распределение
\begin{equation}\label{eq:Uniform}
\centering
U(x, -\sqrt{3}, \sqrt{3})  = 
\begin{cases}
\frac{1}{2\sqrt3}, &\mbox{при } |x| \leq \sqrt3 \\ 0 , &\mbox{при } |x| \textgreater \sqrt3
\end{cases}
\end{equation}
\end{item}
\end{itemize}

\subsection{Характеристики положения}

\begin{itemize}
\begin{item}
Выборочное среднее
\begin{equation}\label{eq:mean}
\centering
\overline x =\frac{1}{n}\sum_{i=1}^n x_i
\end{equation}
\end{item}

\begin{item}
Выборочная медиана
\begin{equation}\label{eq:med}
\centering
med x =
\begin{cases}
x_{(l + 1)} &\mbox{при  $n = 2l + 1$} \\ \frac{x_{(l)} + x_{(l + 1)}}{2} &\mbox{при  $n = 2l$}
\end{cases}
\end{equation}
\end{item}

\begin{item}
Полусумма экстремальных выборочных элементов
\begin{equation}\label{eq:semisum}
\centering
z_R = \frac{x_{(1)} + x_{(n)}}{2}
\end{equation}
\end{item}

\begin{item}
Полусумма квартилей \\
Выборочная квартиль $z_p$ порядка $p$ определяется формулой
\begin{equation}\label{eq:quartile}
\centering
z_p =
\begin{cases}
x_{([np] + 1)} &\mbox{при  $np$ дробном} \\ x_{(np)} &\mbox{при  $np$ целом}
\end{cases}
\end{equation}
Полусумма квартилей
\begin{equation}\label{eq:semiquartile}
\centering
z_Q = \frac{z_{1/4} + z_{3/4}}{2}
\end{equation}

\end{item}

\begin{item}
Усечённое среднее
\begin{equation}\label{eq:trim}
\centering
z_R = \frac{1}{n - 2r} \sum_{i=r+1}^{n - r} x_{(i)}, r \approx \frac{n}{4}
\end{equation}
\end{item}

\end{itemize}

\subsection{Характеристики рассеяния}
Выборочная дисперсия

\begin{equation}\label{eq:samplevariance}
\centering
D = \frac{1}{n} \sum_{i=1}^{n} x_i - \overline x
\end{equation}

\section {Реализация}
Лабораторная работа выполнена с помощью встроенных средств языка программирования Python в среде разработки Jupyter Notebook и Visual Code. Исходный код лабораторной
работы приведён в приложении.

\section {Результаты}

\begin{table}[H]
  \centering
  \begin{tabular}{ | c | c | c | c | c | c | c | }
	\hline
	Normal n = 10 & & & & &  \\ \hline
	& $\overline x$~\eqref{eq:mean} & $med x$~\eqref{eq:mean}& $z_R $~\eqref{eq:semisum} & $z_Q $~\eqref{eq:semiquartile}  &  $z_{tr}$~\eqref{eq:trim}  \\ \hline
	$E(z) ~\eqref{eq:expectation}$ & -0.01240 & -0.02489 & -0.00474 & -0.01261 & -0.44611 \\ \hline
	$D(z) ~\eqref{eq:dispertion}$ & 0.09935 & 0.13874 & 0.18181 & 0.11633 & 0.19166 \\ \hline
	
	Normal n = 100 & & & & &  \\ \hline
	& $\overline x$ & $med x$& $z_R $& $z_Q $&  $z_{tr}$\\ \hline
	$E(z)$ & -0.00618 & -0.00857 & -0.00707 & 0.00268 & -0.53702 \\ \hline
	$D(z)$ & 0.05471 & 0.07758 & 0.13405 & 0.06471 & 0.11674 \\ \hline
	
	Normal n = 1000 & & & & &  \\ \hline
	& $\overline x$ & $med x$& $z_R $& $z_Q $&  $z_{tr}$\\ \hline
	$E(z)$ & -0.00409 & -0.00551 & -0.00393 & 0.00217 &-0.56976 \\ \hline
	$D(z)$ & 0.03683 & 0.05231 & 0.11162 & 0.04357 & 0.08085 \\ \hline
	\end{tabular}
  \label{table:normal_table}
\caption{Нормальное распределение}
\end{table}


\begin{table}[H]
  \centering
  \begin{tabular}{ | c | c | c | c | c | c | c | }
	\hline
	Cauchy n = 10 & & & & &  \\ \hline
         & $\overline x$& $med x$& $z_R $ & $z_Q $  &  $z_{tr}$  \\ \hline
         $E(z)$ & 4.65358 & 0.02889 & 23.04936 & 0.03492 & -4.33522 \\ \hline
         $D(z)$ & 15908.09147 & 0.43148 & 397531.77972 & 1.26249 & 523.04326 \\ \hline

Cauchy n = 100 & & & & &  \\ \hline
         & $\overline x$& $med x$& $z_R $ & $z_Q $  &  $z_{tr}$  \\ \hline
         $E(z)$ & 1.72418 & 0.01434 & -18.42564 & 0.03585 & -7.08401 \\ \hline
         $D(z)$ & 8637.35792 & 0.22896 & 1880864.30928 & 0.65602 & 2509.25432 \\ \hline

Cauchy n = 1000 & & & & &  \\ \hline
         & $\overline x$& $med x$& $z_R $ & $z_Q $  &  $z_{tr}$  \\ \hline
         $E(z)$ & 1.55612 & 0.01011 & 190.09045 & 0.02555 & -6.98385 \\ \hline
         $D(z)$ & 6106.00336 & 0.15343 & 87773120.69259 & 0.43918 & 1712.09914 \\ \hline
	\end{tabular}
  \label{table:cauchy_table}
\caption{Распределение Коши}
\end{table}

\begin{table}[H]
  \centering
  \begin{tabular}{ | c | c | c | c | c | c | c | }
	\hline
Laplace n = 10 & & & & &  \\ \hline
         & $\overline x$& $med x$& $z_R $ & $z_Q $  &  $z_{tr}$  \\ \hline
         $E(z)$ & 0.00358 & -0.00151 & 0.02326 & 0.00012 & -0.40006 \\ \hline
         $D(z)$ & 0.10735 & 0.08220 & 0.41080 & 0.11063 & 0.18386 \\ \hline

Laplace n = 100 & & & & &  \\ \hline
         & $\overline x$& $med x$& $z_R $ & $z_Q $  &  $z_{tr}$  \\ \hline
         $E(z)$ & 0.00153 & -0.00143 & 0.04186 & 0.00484 & -0.49776 \\ \hline
         $D(z)$ & 0.05890 & 0.04395 & 0.40008 & 0.06062 & 0.11308 \\ \hline

Laplace n = 1000 & & & & &  \\ \hline
         & $\overline x$& $med x$& $z_R $ & $z_Q $  &  $z_{tr}$  \\ \hline
         $E(z)$ & 0.00126 & -0.00117 & 0.02586 & 0.00361 & -0.53088 \\ \hline
         $D(z)$ & 0.03961 & 0.02948 & 0.40734 & 0.04075 & 0.07833 \\ \hline
	\end{tabular}
  \label{table:laplace_table}
\caption{Распределение Лапласа}
\end{table}

\begin{table}[H]
  \centering
  \begin{tabular}{ | c | c | c | c | c | c | c | }
	\hline
Poisson n = 10 & & & & &  \\ \hline
         & $\overline x$& $med x$& $z_R $ & $z_Q $  &  $z_{tr}$  \\ \hline
         $E(z)$ & 10.03390 & 9.87850 & 10.35950 & 9.96100 & 11.98083 \\ \hline
         $D(z)$ & 0.94092 & 1.30249 & 1.90001 & 1.19398 & 1.61149 \\ \hline

Poisson n = 100 & & & & &  \\ \hline
         & $\overline x$& $med x$& $z_R $ & $z_Q $  &  $z_{tr}$  \\ \hline
         $E(z)$ & 10.00905 & 9.85175 & 10.64650 & 9.95325 & 12.45501 \\ \hline
         $D(z)$ & 0.51966 & 0.74990 & 1.51329 & 0.67369 & 1.14114 \\ \hline

Poisson n = 1000 & & & & &  \\ \hline
         & $\overline x$& $med x$& $z_R $ & $z_Q $  &  $z_{tr}$  \\ \hline
         $E(z)$ & 10.00531 & 9.89917 & 10.97383 & 9.96717 & 12.61537 \\ \hline
         $D(z)$ & 0.34978 & 0.50642 & 1.43907 & 0.45051 & 0.81966 \\ \hline
	\end{tabular}
  \label{table:poisson_table}
\caption{Распределение Пуассона}
\end{table}

\begin{table}[H]
  \centering
  \begin{tabular}{ | c | c | c | c | c | c | c | }
	\hline
Uniform n = 10 & & & & &  \\ \hline
         & $\overline x$& $med x$& $z_R $ & $z_Q $  &  $z_{tr}$  \\ \hline
         $E(z)$ & 0.00634 & 0.00643 & 0.00743 & 0.00912 & -0.41120 \\ \hline
         $D(z)$ & 0.10099 & 0.23114 & 0.04599 & 0.13631 & 0.22124 \\ \hline

Uniform n = 100 & & & & &  \\ \hline
         & $\overline x$& $med x$& $z_R $ & $z_Q $  &  $z_{tr}$  \\ \hline
         $E(z)$ & -0.00131 & -0.00276 & 0.00308 & 0.00859 & -0.53424 \\ \hline
         $D(z)$ & 0.05538 & 0.13012 & 0.02329 & 0.07517 & 0.13984 \\ \hline

Uniform n = 1000 & & & & &  \\ \hline
         & $\overline x$& $med x$& $z_R $ & $z_Q $  &  $z_{tr}$  \\ \hline
         $E(z)$ & -0.00070 & -0.00120 & 0.00207 & 0.00667 & -0.57210 \\ \hline
         $D(z)$ & 0.03726 & 0.08778 & 0.01553 & 0.05062 & 0.09711 \\ \hline
	\end{tabular}
  \label{table:uniform_table}
\caption{Равномерное распределение}
\end{table}

\section{Обсуждение}
Из полученных данных видно, что среднее~\eqref{eq:expectation} всех характеристик стремится к теоретическому, а оценка дисперсии~\eqref{eq:dispertion} к нулю при увеличении размера выборки. В случае распределения Коши~\eqref{eq:Cauchy} это верно только для характеристик положения.

\section{Приложения}
Репозиторий на GitHub с релизацией: \href{https://github.com/WiillyWonka/MatStat}{github.com}.
\end{document}
